\documentclass[12pt]{article}	
\usepackage[T1]{fontenc}
\usepackage{fourier}

%\usepackage[english]{babel}															% English language/hyphenation
%\usepackage[protrusion=true,expansion=true]{microtype}				% Better typography

\usepackage{amsmath,amsfonts,amsthm}										% Math packages
\usepackage{graphicx}	
\usepackage{natbib}													% Enable pdflatex
\usepackage{hyperref}

\hypersetup{
    colorlinks=true,       % false: boxed links; true: colored links
    linkcolor=red,          % color of internal links (change box color with linkbordercolor)
    citecolor=black,        % color of links to bibliography
    filecolor=magenta,      % color of file links
    urlcolor=blue           % color of external links
}

\renewcommand{\baselinestretch}{1.2}
% skip a line between paragraphs, no indentation
\setlength{\parskip}{1.5ex plus0.5ex minus0.5ex}
\setlength{\parindent}{0pt}

%%% Custom sectioning (sectsty package)
%\usepackage{sectsty}												% Custom sectioning (see below)
%\allsectionsfont{\centering \normalfont\scshape}	% Change font of al section commands


%%% Custom headers/footers (fancyhdr package)
\usepackage{fancyhdr}
\pagestyle{fancyplain}
\fancyhead{}		
\fancyhead[L]{ }
\fancyhead[R]{ \includegraphics[width=1.5cm]{qe-logo.png} }
\fancyfoot[L]{}	% You may remove/edit this line 
\fancyfoot[C]{}													% Empty
\fancyfoot[R]{\thepage}									% Pagenumbering
\renewcommand{\headrulewidth}{0.3pt}			% Remove header underlines
\renewcommand{\footrulewidth}{0.3pt}				% Remove footer underlines
\setlength{\headheight}{28pt}
\setlength{\footskip}{28pt}
% 
%\usepackage{enumitem}
%\setlist[enumerate]{itemsep=0pt,topsep=2pt}
%\setlist[itemize]{itemsep=0pt,topsep=2pt}
%\setlist[enumerate,1]{label=(\roman*)}


%%% Equation and float numbering
\numberwithin{equation}{section}		% Equationnumbering: section.eq#
%\numberwithin{figure}{section}			% Figurenumbering: section.fig#
\numberwithin{table}{section}				% Tablenumbering: section.tab#


%%% Maketitle metadata
\newcommand{\horrule}[1]{\rule{\linewidth}{#1}} 	% Horizontal rule

\title{
		%\vspace{-1in} 	
		\usefont{OT1}{bch}{b}{n}
		\normalfont \normalsize \textsc{Summer Course for Economics Students and Policy Professionals} \\ [25pt]
		\horrule{1pt} \\[0.4cm]
		\huge QuantEcon-RSE Intensive Course in Computational Modeling \\
        \vspace{1em}
		\large Australian National University\\
		\large December 2019\\
		\horrule{1pt} \\[0.5cm]
}
\author{
		\normalfont \normalsize
        Co-Organizers: John Stachurski and Sebastian Wende \\[-3pt]		\normalsize
        %\today
}
\date{}


%%% Begin document
\begin{document}

\maketitle




\begin{center}
    \vspace{-2em}
    \includegraphics[width=2cm]{anu-logo-2.png}
    \vspace{1em}

    \begin{tabular}{ l  r r l}
        \textbf{Dates} & & &  16th--20th December, 2019\\
        \textbf{Location} & & &  Canberra \\
        \textbf{Primary Sponsor} & & &  Research School of Economics \\
        \textbf{Additional Sponsors} & & &  TBA \\
        \textbf{Academic Contact} & & & \texttt{<john.stachurski@anu.edu.au>} \\
        \textbf{Administrative Contact} & & & \texttt{<nicole.millar@anu.edu.au>} \\
    \end{tabular}
\end{center}

\bigskip
\bigskip

\noindent 
This one week summer course for advanced undergraduate students and policy
professionals will provide training in cutting edge computational modeling for
economic analysis.  Intake is highly selective and all university students
accepted to the program will be fully funded.  The course will also provide a
forum for students to interact with policy professionals from the Treasury,
the Productivity Commission and other leading
institutions.

The motivation for this short course is that computationally
intensive models are increasingly being used for applied policy analysis.
As a result, economic modeling now requires strong computational skills.  For example, many
computational models in economics are built on top of an optimal choice
framework called dynamic programming, which is used in contexts such as fiscal
and monetary policy, as well as analyses of industry-specific policy including
water in the Murray-Darling Basin \citep{Grafton2011} and energy demand and supply \citep{Ringkjob2018}.  State-of-the-art dynamic programming is computationally intensive and
     intellectually demanding.

This one-week intensive course will establish a baseline competency in
those computational methods required to apply dynamic programming and other
related techniques for the purpose of public
policy analysis. The skills gained will be
directly useful for public policy practitioners. For students, the course will
provide a basis for further study and research, a chance to meet and discuss
in-demand skills with policy professionals, and a demonstrable
applied skill for prospective job applications. It will build the
pool of talent available to public policy institutions requiring computational
economic modeling.



\subsection*{Goals}

The goal of this course is to 
%
\begin{enumerate}
    \item develop capacity in computational modelling for the analysis of Australian economic data,
    \item grow interest in computational economics in order to encourage students to
        undertake further research and learning in the area, and 
    \item build bridges between academic economists, students and Australia's
        leading policy institutions.
\end{enumerate}



\subsection*{Instructors}

The primary instructor is John Stachurski.  Guest lecturers will include
Dr.\ Fedor Iskhakov (ANU) and policy professionals (TBA).


\subsection*{Structure}

The structure of the course for its one week duration will be as follows:

\begin{itemize}
    \item 9am-12pm: Lectures
    \item 1pm-5pm: Group exercises 
\end{itemize}

There will also be a social event sponsored by the Research School of
Economics, as well as presentations from representatives of leading policy
institutions on economic modeling in practice.


\subsection*{Course Content}

The coding language of instruction will be Python, which is rapidly rising
in popularity for scientific computing, artificial intelligence and machine
learning.  The course will cover the following topics:

\begin{itemize}
    \item Python and its scientific libraries
    \item Dynamic models and distributions
    \item Dynamic programming and optimization
    \item Basic principles of software engineering
    \item Introduction to high performance computing
\end{itemize}

Applications will be drawn from macroeconomics, finance and applied
microeconomics.

\subsection*{Financial Support}

The course is fully funded. Not only will course costs be covered for students but accomodation and travel cost will be covered for students from outside of Canberra.

\subsection*{Prerequisite Knowledge}

Attendees will require familiarity with calculus and linear algebra.
All attendees should have a strong interest in computing.

\subsection*{Applications}

To apply studnets should write up to a one page pitch detailing why they would benefit from the course. A screenshot showing competetion of a \href{https://lectures.quantecon.org/py/}{quantecon lecture} would demonstrate both prerequist skills and interest. 
Applications should be sent to both \href{mailto:john.stachurski@gmail.com}{john.stachurski@gmail.com} and \href{mailto:sebastian.wende@anu.edu.au}{sebastian.wende@anu.edu.au}. Students should also attach an academic transcript to their applciation email.

\bibliography{bib}

\end{document}
